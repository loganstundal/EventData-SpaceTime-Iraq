\documentclass[12pt]{article}
\usepackage{graphicx}
\usepackage{pstricks}
\usepackage[letterpaper,pdftex,right=1in,left=1in,top=1in,bottom=1in]{geometry}
\usepackage{amsmath}
\usepackage{amsthm}
\usepackage{amssymb}
\usepackage{setspace}
\usepackage{multirow}
\usepackage{hyperref}
\renewcommand{\bibitem}{\vskip
2pt\par\hangindent\parindent\hskip-\parindent}
\renewcommand{\baselinestretch}{2.0}
\title{Geostatistical Analysis in Space and Time}

\author{Logan Stundal, Benjamin Bagozzi, and John Freeman}
\begin{document}
\maketitle
\doublespace

Spatial temporal processes are at the heart of many important political questions. How do
ideologies of mass publics evolve within and across state boundaries.
How does electronic media affect the attitudes and behavior of citizens over time
within counties in different states, attitudes and behaviors which through political
communication diffuse to and from neighboring counties? How do candidates decide where and when to
to raise funds ("search for gold") in their
constituencies? How do their efforts spawn new contributions over time
from locations outside their constituencies? Why and how do civil wars break out in and then spread across
governmental jurisdictions dying out in some places but escalating in others?

Political methodology lags behind other disciplines like criminology, meteorology, environmental
science, and epidemeology in its analysis of spatial temporal processes. We made
real advances in spatial measurement (Monogan and Gill 2015; Cho 2003), causal inference based
on geographical designs (Keele and Titiunik 2015) and the study of conflict
dynamics (Brandt, et al 2008, 2014). But, many of our spatial analysis ignore or
downplay temporal factors; they assess treatment effects at a single point in time. Or, we
study spatial processes but in separate or highly aggregated slices of time (Cho and Gimpel 2007).
Leading textbooks on time series in political science and economics
(Box Steffensmeier et al. 2014, Enders 2010) say nothing about spatial-temporal
processes. Difference in difference causal analysis use two time points to
make grand assumptions about time trends in
complex political process without any provision for medium and long term dynamics, \
let alone diffusion across (omitted) units of analysis (Keele and Minozzi 2013).
Many conflict studies are macro level investigations in which spatial factors
amount to country dyadic exchanges or exchanges between  belligerants (Brandt et al. 2008). Subnational
conflict studies make a "localization" assumptions--that conflict occurs over time within
but not across units of analysis (Silverman 2021).

If they allow for spatial temporal factors in their analysis, political scientists most
often place two way fixed effects in their models and, concomitantly,used cluster errors
to account for any remaining unit heteroscedasticity. Occasionally, a
simple (static) spatial model is used to assess the robustness of the
investigator's results. As we explain below, conceptually, two way fixed effects models
do not allow for diffusion of the effects of the variables of interest across
units and they treat the effect of "time" as identical on all units in each
time slice; two way fixed effects set-ups ignore the possibility of individual
space-time unit differences in the effects of causal mechanisms.

Some of us have applied and made advances in neighborhood type spatial temporal
modeling. These modelers carve up space into discrete units and include in regression
models right hand side variables with both spatial lags with \emph{prespecified}
connectivity matrices of the dependent variable and also
temporal lags of the values of the own unit's past values
of the dependent variable. The development of such models for binary dependent
variables along with techniques to assess spatial and temporal marginal effects
is illustrative (Ward and Gleditsch 2002, Franzese et al. 2015). An innovative application
of spatial vector autoregression in the study of subnational conflict also has
been applied in international relations (Linke et al. 2012).

With one or two exceptions, what has not been applied and developed in political science,
is the  geostatistical approach to spatial temporal analysis. This approach treats
the data generating process as continuous in space; it employs a set of techniques
for "point process" modeling. Rather prespecifying a connectivity matrix, the
geoestatistical approach \emph{estimates} the extent of spatial interdependence
between units and allows for unit variant temporal evolution of causal mechanisms
like diffusion. Out of the geostatistical approach emerges methods for tracking
\textit {the distinct effects of causal mechanisms for every individual unit at each point
in time.} Its is this methodology that has proven so useful recently in other
disciplines.\footnote{Geostatistical modeling is mentioned as a possible
approach in one paragraph at the end of the second edition of Ward and Gleditsch
(2019). And it has been applied in the form of kriging by
by Monogan and Gill 2015, Gill 2020, Cho and Gimpel 2007. But these studies are
static in character. The spatial-temporal versions of kriging, what is
called, co-kriging, to our knowledge,
have not been applied by American political methodologists. But see
Pavia et al. 2008.}

This research note explains a popular geostatistical approach and compares it to
the more familiar two way fixed effects and neighborhood methods of modeling
spatial temporal processes. We begin by reviewing the ways two familiar
methods conceive of spatial temporal processes. The key idea of
separability is introduced in this context. Geostatistical modeling based
on a stochastic partial differential equation then is introduced.
A formal definition of separability and the additional concepts of
space-time stationarity and isotropy are discussed briefly here.
 We then the compare the three approaches. We
use the modeling of insurgent attacks in Iraq--in particular, Silverman's
recent (2021) two way fixed effects based investigation--as our benchmark.
As an added payoff of our analysis, we compare the modeling results for
ground truth (army reported SIGACTS), machine coded (ICEWS) and human
coded (GED) databases; the rationale for our comparison of the databases
is explained elsewhere (Bagozzi et al. 2018; Stundal et al. forthcoming).

We find....

\section{Geostatistical Modeling Spatial Temporal Processes}

Spatial temporal modeling is motivated by theoretical expectations predict and empirical observations
which suggest the existence of certain causal mechanisms: as regards space, the possibility
of strategic interaction, diffusion, spillover, relocation, and advection and as regards time, persistence,
cumulation, and memory. Spatial temporal models aim to capture the \emph{combination} of
such mechanisms. The notion that civil conflict in a particular location in one part of country not only
spills over to another part of the same or neighboring countries contemporaneously and
historically at the same time the conflict persists (is remembered) within the unit in
which it originated is illustrative. Schutte and Weidman (2011) are among those who explain why
spatial-temporal interdependence should be expected in civil conflicts. They detect evidence
of "escalation diffusion" in the Bosnian, Burundi, and Rwandan cases.

Two way fixed effects set ups are among the most common approaches political scientists use
to study spatial temporal political processes. Assume we observe $i = 1 \dots N$ units at $t= 1 \dots T$ points
in time. And we are interested in the effect of a set of covariates, $X_{kit}$ on variable $Y_{it}$.
This most simple model is of the form
\begin{equation}
Y_{it} = \alpha_i + \gamma_t + \beta X_{it} + \epsilon_t
\end{equation}

where $Y_{it}$ is a Nx1 vector of dependent variables, $X_{kit}$ is a Nxk matrix of
covariates, $\alpha_i$ is the fixed (time invariant) affect of factors at each location i,
$\gamma_t$ is the fixed (unit invariant) affect of unknown factors at each respective
time point, $\beta$ is a coefficient to be estimated and $\epsilon_{it}$ is a Nx1 error
vector which may be heteroscedastic. The fixed effects are separated and added together.
There is no provision for interdependence between the values of $Y_{it}$ of any units,
there is no persistence or other kinds of temporal mechanisms governing the evolution
of $Y_{it}$, and there is no variation in the impact of historical factors across units at each
point in time.\footnote{Other simple models are used in political science such as the unit fixed
effects model with a lagged outcome variable, $Y_{it}= \alpha_i + \beta X_{kit} + \rho Y_{i, t-1} +
\epsilon_{it}$. And more complete representations of the effect of history are often employed
such as time polynomials. But, again, these alternative set ups ignore the possibility of
spatial interdependence and treat the effect of history as unit invariant. On the latter
point see the discussion Franzese et al. 2015 esp. pps. 152 and 165.}

A second, familiar model is the so called Spatial Temporal Autoregressive (STAR) Neighborhood
model. These are usually expressed as

\begin{equation}
Y_{it} = \rho W Y_{t} + \phi M Y_{i,t-1} + \beta X_{kit} + \epsilon_{it}
\end{equation}

As Franzese et al. (2007: 159) explain, W is the
Kronecket product of a TxT identity matrix and an NxN prespecified spatial weights matrix
$(I_t \bigotimes W_N)$, M is an NTxNT matrix of zeros except for ones on the minor
diagonal at coordinates (N+1,1), (N+2,2) ...(NT,NT-N), and $\phi$ is the unit invariant
temporal autoregressive coefficient. Anselin (2006: Section 26.2.1)) argues that this set-up captures
diffusion, persistence and the other causal mechansms enumerated above. The first term on the right represents
spatial interdependence while the second term captures unit persistence and related
temporal factors. Again, these factors are assumed to be separable and additive. Methods
\footnote
{There is a related model that according to Anselin and others is best suited to
analyzing spatial connections in the errors.Stundal et al. (2021) argue that conceptually, the
spatial error model (SEM)for answering our questions about
the validity of human and machine coded event data. SEMs capture model errors for
neighboring units that cluster together---``smaller(larger) errors for observation $i \ldots$
go together with smaller [larger] errors for [neighbor] $j$" (Ward and Gleditsch 2019: 76).
Errors also may be correlated because of the mismatch between the spatial scale of a process
and the discrete spatial units of observations (Anselin 2006: 907). These error patterns
correspond respectively to what researchers call the remoteness problem. For example,
remoteness means that a model's underestimates of violence in a unit distant from a
capital city correlate with underestimates of violence in a neighboring unit which is
also distant from the same city. An example, is the spatial probit error model, SPEM.
A technique based on the conditional log likelihood and variance-covariance matrix of
the model can be used to estimate it (Martinelli and Geniaux 2017). The model provides
estimates of a $\lambda$ parameter which, with a row-standardized connectivity matrix,
indicates the average dependence in the errors of a prespecified set of neighbors on
the estimation error in a unit of interest.}

 Neighborhood models may suffer from ``inappropriate discretization" (Lindgren and Rue 2015: 3).
 The prespecified connectivity matrices used in neighborhood models treats spatial dependence
 as a step function---the same for some subset of units and nonexistant for another subset of
 units. The same is true of methods used to detect spatial-temporal interdependence such as
 that developed by Schutte and Weidman (2011) who use a "Moore neighborhood" to assess
 escalation diffusion and relocation. In reality the spatial dependence may vary \textit{continuously}
 in space.  The alternative---geostatistical spatial error models---produce \textit{estimates}
 of both the range of spatial error dependence and of the site specific impact of unobserved
 factors.

Geostatistical models analyze point referenced data. These models are based on the idea of a
continuous spatial domain. For example, even though terrorist events are observed at specific
locations and therefore ``inherently discrete" these events are interpreted as realizations of
 a continuously indexed space-time process of violence perpetrated against civilians and
 government officials (Python et al. 2017, 2018). Formally, the data are defined by a
 process indexed in time and space:
$Y(s,t) \equiv [y(s,t), (s,t) \in D \subset R^2 x R)$. The spatial-temporal covariance function
for the process is written as $Cov(y(s_i,t)y(s_j,u)) = C(y_{it},y_{ju})$. Under an assumption
of stationarity in space and time (see below) the covariance function can be expressed in terms of a combination
of spatial distance, $\Delta_{ij} = ||s_i - s_j||$, and temporal lag $\Lambda_{tu} = |t-u|$.
Thus $Cov(y_{i},y_{ju}) =
 C(\Delta_{ij}, \Lambda_{tu}).$ \footnote{As regards the spatial location of the data,
 $s$, can vary continuously in a fixed domain;
 $s$ is a two-dimensional vector with latitudes and longitudes (three-dimensional if
 altitudes are considered) This definition of the process is taken from Blangiardo
 and Cameletti 2015: 235-236.}

As in neighborhood modeling, several concepts are especially important in geostatistical
analysis. First, as in the fixed effects models and STAR setups, spatial and temporal
factors often are assumed to be \emph{separable}. This means that covariance function can be written as a
sum or product of its spatial and temporal parts,
for instance, $Cov(y_{it},y_{ju}) = C_1(\Delta_{ij})C_2(\Lambda_{tu})$. Gneiting (2002) proposes
a test for separability for modeling spatial-temporal data.\footnote{Harvill (2010: 375) explains
that ``covariance functions imply that small changes in the locations of
observations can lead to large changes in the correlations between certain linear combinations
of observations."}
\textit{Stationarity} is a second key concept. Stationarity in space implies that mean function of
the process is constant in space and that the spatial covariance function depends only on the
distance vector $s_i - s_j \in R $. A closely related concept is \emph{isotropy}; this is the
idea that the covariance function depends only on
Euclidian distance, $||s_i - s_j|| \in R$ (rotational invariance). For instance,
 Bandyopadhyay and Rao (2017) provide a test for
stationarity for irregularly spaced data.  In time series analysis stationarity means, roughly, that the mean and
all its autocovariances are unaffected by a change in its time origin. The Dickey Fuller
test is an example of a test for nonstationarity.  Because time is ordered and space is not,
isotropy has no meaning in the spatial-temporal context (Harvill 2010: 375).\footnote{Technically
this definition of spatial stationarity is a definition of second order stationarity.
``Formally, the spatial process is
a Gaussian field if for any $n \geq 1$ and for each set of locations $(s_1, \ldots, s_n)$ the
vector $(y(s_1), \dots, y(s_n)$ follows a multivariate Normal distribution with mean
$\mu = (\mu(s_1) \ldots, \mu(s_n)$  and spatially structured covariance matrix $\Sigma$.
[Then in terms of the notation in the text above] the spatial process is second order
stationary if the mean function is constant in space, i.e., $\mu(s_i) = \mu$ for each i and
the spatial covariance function depends only on the distance vector $(s_i- s_j) \in R^2$"
(Blangiardo and Cameletti 2015: 193. In atmospheric
studies second order stationarity might be violated due to togological factors. Jun and
Cook (2020) propose to model
terrorism and related phenomenon as nonstationary processes. As regards the temporal
part of the process, let $y_t$ be a time series. Then this series is \textit{covariance
stationary} if for all t and t-s,
\begin{equation}
E(y_t) = E(y_{t-s}) = \mu
\end{equation}
\begin{equation}
E(y_t - \mu)^2 = E[(y_{t-s} - \mu)^2] = \sigma_{y}^2 \\, \hspace{1cm} var(y_t) = var (y_{t-s}) = \sigma_{y}^2
\end{equation}
\begin{equation}
E[(y_t - \mu)(y_{t-s} - \mu)] = E[(y_{t-j} - \mu)(y_{t-j-s} - \mu)] = \gamma_s \\
\end{equation}
where $\mu, \sigma_{y}^2$, and  $\gamma_s$ are all constants (Enders 2010: 64). Financial
time series and, according to some scholars, approval of the President, are
nonstationary.}


A popular geostatistical approach is Continuous
Domain Bayesian Modeling with Integrated, Nested Laplacian Approximation
(INLA).\footnote{The following description draws from Blangiardo and Cameletti (2015: Chapter 6)
and especially the passage on pps. 234-5 of Python et al.\ (2017). Extensions that allow for
modeling nonstationary processes are
reviewed Ingebritsen  et al. (2015). } Briefly, this approach ``does not build models solely
for \emph{discretely observed data} but for approximations of \emph{entire processes} defined
over continuous domains" (Lindgren and Rue 2015:3, emphasis in the original). It assumes that
 the data generating process is a Gaussian field, $\xi(s)$, where $s$ denotes a finite set of
 locations, $(s_1 \ldots s_m)$. As such it suffers from a ``big n problem;" analyzing the
 Gaussian field is costly computationally (Lindgren et al. 2011). Therefore, a particular
 linear, stochastic partial differential equation is assumed to apply to the Gaussian field:

\begin{equation}
  (\kappa^2 - \Delta)^{\frac{\alpha}{2}}(\tau \xi (s)) = W(s), \hspace*{1cm} s \in D
\end{equation}
\noindent where $\Delta$ is a Laplacian,  $\alpha$ is a smoothness parameter such that
$\alpha = \lambda + 1$ (for two dimensional processes), $\kappa > 0$ is a scale parameter,
$\tau$ is a precision parameter, the domain is denoted by $D$, and $W(s)$ is Gaussian spatial
 white noise. The stationary solution to this equation is a Gaussian field with the
 Mat\'{e}rn covariance function:

\begin{equation}
  Cov(\xi(s_i),\xi(s_j)) = \sigma^2_{\xi_i}\frac{1}{\Gamma(\lambda)2^{\lambda-1}}(\kappa \mid\mid s_i - s_j \mid\mid)^\lambda K_\lambda(\kappa \mid\mid s_i - s_j \mid\mid)
\end{equation}
\noindent where $\mid\mid s_i -s_j\mid\mid$ denotes the Euclidean distance between locations $s_i$ and $s_j$, $\sigma^2_{\xi_i}$ is the marginal variance, $\Gamma(\lambda) = \lambda !$, $K_\lambda$ is the modified Bessel function of the second kind and order $\lambda > 0$. The distance at which the spatial correlation becomes negligible (for $\lambda > .05)$ is the range, $r$. The solution to the SPDE implies that the formula for the marginal variance is $\sigma^2 = \frac{\Gamma(\lambda)}{\Gamma(\alpha)(4\pi)^{\frac{d}{2}}\kappa^{2\lambda}\tau^2}$ where $d=2(\alpha - \lambda)$. And the formula for the range is $r = \frac{\sqrt{8\lambda}}{\kappa}$. In this way, the Gaussian field can be represented (approximated) by a GMRF. A finite element method using basis functions defined on a Constrained Refined Delaunay Triangularization (mesh) over a corresponding shapefile of latitude-longitude event data is used for this purpose.

A hierarchical Bayesian framework then can be used to model one's data. For the variable of interest in our application below---insurgent violence incidents (V)
normalized by unit population---we employ:
\begin{equation}
V(s_i,t)|\theta_V,\zeta_V(s_i,t) \sim Gaussian [\mu_V(s_i,t)]
\end{equation}
\begin{equation}
\mu_V(s_i,t) = \beta_{V0} + x_V(s_i,t)\beta_V + \zeta_V(s_i,t) + \epsilon_V(s_i,t)
\end{equation}
\begin{equation}
  \theta_V \sim p(\theta_V)
\end{equation}
where $V(s_i,t)$ is the observation at point $s_i$ at time t, $x_V$ is a vector of
covariates, $x_V(s_i,t)=(x_{V1}(s_i,t) \ldots, x_{Vq}(s_i,t))$, $\beta_V$ is the corresponding
vector of coefficients, $\beta_V = (\beta_{V1}\ \ldots \beta_{Vq})'$, $\zeta_V(s_i,t)$ is the Gaussian
Markov Random Field, and $\epsilon_{V}(s_i,t) \sim N(0,\sigma_{\epsilon_{V}}^2)$ is measurement error
with variance $\sigma_{\epsilon_V}^2$. Cameletti et al. (2010) refer to the covariates as (separable) large scale
factors and the $\zeta_V$ as small scale processes. The latter capture diffusion and related, \textit
{unit and time variant} spatial-temporal causal mechanisms.\footnote{Python et al. (2018: 17) say
the space time structure of the GMRF captures, in the form of their ``aggregate effects,"
"the marginal effects of unobserved individual factors."
Cameletti et al. (2010), Harvill (2010) and others interpret the GMRF as small scale factors
manifesting diffusion, spillover, advection and other causal processes.}
The Gaussian Markov Random field
aometimes is assumed to follow a first or second order random walk (Python et. al. 2018). For the first order random walk
we have $\zeta_{V}(s_i,t)|\zeta_{V}(s_{i,t-1}) \sim N(\zeta_{V}(s_{i,t-1},\sigma_{rw}^2))$ so
that $cov(\zeta_{V}(s_i,t),\zeta_{V}(s_j,u)) = 0$ if $t\ne u$ and the spatial interdependence is
$cov(\zeta_{V}(s_i),\zeta_{V}(s_j))$ if t=u. Modeled in this way, the temporal dependence amounts
to simple, \emph{unit variant}, one lag persistence in the mean of the GMRF (not temporal dependence on more distant
lags of $\zeta_V(s_i,t)$ nor on lagged values of the $\zeta_V(s_j,t)$ at other locations).
In other cases (Python et al. 2016), the the GMRF is assumed to follow a first order autoregressive
process: $\zeta(s_i,t) = \rho \zeta(s_i,t-1) + \psi(s_i,t)$ where $\psi(s_i,t)$ is a time independent
zero mean Gaussian field with $Cov(\zeta(s_i,t),\zeta(s_j,u)) = 0$ if $t \ne u$ and $Cov(\zeta(s_i)\zeta(s_j)$
if t = u. We use the latter specification in our analysis below.\footnote{This notation follows
that used by Python et al. (2018):7-9.}\footnote{A more complex, dynamic set up in which the
weights on the bases functions of the spatial representation are governed by a linear
dynamic system is developed by Cseke et al. (2016) and used to analyze data from the Afghan
War Diary.}

Besides estimates of the effects of the covariates on the rate of violent events, this geostatistical
model produces useful estimates of the parameters in the GMRF---in particular, the mean and
standard deviation of the latent field at each point in space and at each poin in time.
These estimates thus tell us not just that spatial-temporal interdependence exists as
in the Schutte and Weidman (2011) method, but \emph{where and when} such interdependence
exists.


Table 1 summarizes these three (separable) spatial temporal models along with some others
that have been used to study civil conflict.

\begin{table}
\centering
\begin{tabular}{|c|c|}
\hline Neighborhood, Discrete Space Time & Geostatistical, Continuous Space Time\\
\hline
 & \\
-General Linear Models with Two Way Fixed Effects  & -Stochastic Integral Differential\\
and Error Clustering & Equation Models (square lattice)\\
 Condra and Shapiro 2015 & Zammit-Mangion 2012 \\
 Berman et al. 2015 & \\
 Weidman and Shapiro 2015 & \\
 Blair, 2021; Silverman 2021 & \\
  & \\
-Spatial Temporal Autoregressive & -Stochastic Partial Differential\\
Models & Equation Models (triangular mesh) \\
 Weidman and Ward 2010 & Python et al. 2016,2018 \\
 Franzese et al. 2015 & \\
 & \\
-Spatial Vector Autoregressions & -State Space Models \\
Linke et al. 2012 & Cseke et al. 2016\\
 & \\
-Spatial Hierachical & \\
Linear Models & \\
Hagan et al. 2016 & \\
\hline
\end{tabular}
\caption{Types of Separable Space Time Models Used in the Study of Subnational Civil Conflict and Terrorism
with Selected Citations}
\end{table}

\section{Three Approaches to Modeling Spatial-temporal Processes Compared}
Much progress has been made in recent years in analyzing the subnational,
microdynamics of civil war. Compensation studies are illustrative.
They show that insurgent violence tends to be lower if counterinsurgents compensate
civilians for various kinds of harm the counterinsurgents create. Informational
mechanisms---expressed through "civilian agency"---account for the
relationship between post-harm compensation and violence
reduction; the compensated civilians share key information to counterinsurgents
, information which enable counterinsurgents
to thwart subsequent violence. In this way, compensation produces \emph{de-escalation
diffusion} and(or) \emph{relocation de-escalation}. As the citations Table 1 indicates,
(single equation) fixed effects models usually are
employed to establish this relationship. Iraq often is the testbed for the analysis.\footnote
{Authors like Condra and Shapiro (2012), Silverman (2021) and others go to great lengths to establish
the exogeneity of compensatory variables. We do not address their efforts in this
regard but rather focus on the way in which spatial-temporal mechanisms are treated
in the compensation genre.}

An important, recent study in this genre is Silverman's article in the 2021 volume
of \textit{International Organization.} Silverman's main model is:
\begin{equation}
Y_{i,j}-Y_{i,t-1} = \alpha (c_{it}-c_{i,t-1}) + \beta (d_{it}-d_{i,t-1}) + \gamma( e_{i,j}-e_{i,j-1})\\
 + G_y + H_i + I_{g,y} + \mu_{i,j}
\end{equation}

where $Y_{it}$ is the level of insurgent attacks in Iraq district i at time t, $c_{it}$
is the spending on condolence payments in Iraq district i at time t, $d_{it}$ is the
spending on supplemental form of condolence payments, Ruzicka parments, in Iraq district
i at time t, $e_{it}$ is a vector of time varying controls including the amount of collateral
damage created by Coalition forces and insurgents, $G_y$ is a set of half year fixed effects,
$H_i$ is a set of district fixed effects, and $I_{gy}$ is a set of interactions between
governorate level vote shares for Sunni parties  in the 2005 parliamentary elections and
half years intended to pick up ``broad sectarian shifts such as the Sunni Awakening."
Silverman uses the Army's SIGACTS-III database to measure violence; he estimates the
model using OLS for the period February 2004-December 2019 using clustered errors
and with and without population weights. He assumes conflict is "localized" and that
it evolves strictly as a function of changes in the covariates; he makes no
provision for unit specific changes in the rate of violence due diffusion
nor any unit specific persistence of violence in time.\footnote{The sources of Silverman's
condolence payments in the DoD's Commanders Emergency Response Program (CERP) and
the USAID's Maria Ruzicka Iraq War Victims Fund. Population data are taken from the
World Food Programme database. His controls include coalition troop strength,
population density, per cent urban and other variables. See his article for
the sources of these data (2021: 9ff).}

The first column of Table 2 replicates Silverman's main results. Most important
it shows that compensation payments reduce the rate of violence in Iraq in his
study period. \emph{Should we replicate Silverman's analysis using ICEWS and
GED data for his dependent variable?}

----------------------------------------\\
STAR Models. Emphasize Anselin's (2006) argument about causal mechanisms
captured by START model. Note Silverman doesnt analyze data with STAR setup
 Logan's new results for effects of compensation when allowing for unit invariant
 persistence in form of lagged dependent variable. Does this hold up with ICEWS
 and GED? Footnote Blair (2021) use of STAR as robustness checks and flaws
 in his STAR application
 ------------------\\

Geostatistical reanalysis of Silverman. Use of district centroids as points.
Specifications for triangularization and settings for prior distributions, etc.
Admit to temporal discretization ala Cseke et al. 1747 top col 2.
The effects of the localization assumption in Silverman ought to be illuminated
in a way not revealed by the STAR model.

Excerpt for this section:
INLA was implemented in the programming language of R.
R-INLA performs numerical calculation of posterior densities and
in this regard it is more efficient than Markov Chain Monte Carlo methods (Blangiardo and Cameletti,
2015). We use the default priors in R-INLA for the covariate coefficients and measurement
errors, i.e., we assume iid zero-mean normal distributions with precision equal to .001
for the vector $\beta_V$ and $\epsilon_V$. An improper uniform prior between
$-\infty$ and $\infty$ is used for the intercept, $\beta_{V0}$. As regards the
settings for the Gaussian Markov Random Field, ...For its random walk structure
we assume $P(\sigma_{V}^2 > 1 =0.01.$

\section{Statistical Analyses of the External Validity of Machine and Human Coded Data for Iraq}

\newpage
\singlespace
\section{References}

\bibitem Anselin, Luke 2006. "Spatial Econometrics" Chapter 26 in
 \emph{Palgrave Handbook of Econometrics} Volume 1. Econometrics. Basingstone,
 Palgrave MacMillon, pps. 901-969

 \bibitem Bandyopadhyay, Soutir, and S.S. Rao. 2017.
  ``A Test for Stationarity in Irregularly Spaced Data."
  \emph{Journal of the Royal Statistical Society} 79(1): 95-123.

\bibitem Bagozzi, Benjamin E., Patrick T Brandt, John R. Freeman, A, Kim, A. Palao,
  and Carly Potz-Nielsen. 2018 " The Prevalence and Severity of Underreporting Bias
   in Machine and Human Coded Data" \emph{Political Science Research and Methods}
  7(3): 641-649.

\bibitem Berman Eli, Michael Callen, Joseph H. Felter and Jacob Shapiro. 2011.
 "Do Working Men Rebel? Insurgency in Afghanistan, Iraq, and the Philippines."
 \emph{Journal of Conflict Resolution} 55(4): 496-528//

\bibitem Biddle, Stephen, Jeffrey A. Friedman, and Jacob N. Shapiro. 2020.
 "Testing the Surge: Why Did Violence Decline in Iraq in 2007? \emph{International
 Security} 37(1): 7-40.//

\bibitem Blair, Christopher W. 2021 "Restitution or Retribution? Detainee Payments
 and Insurgent Violence" Unpublished Manuscript. State College, PA.

\bibitem Blangiardo, Marta and Michela Cameletti. 2015. \emph{Spatial and
 Spatial-temporal Bayesian Models with R-INLA} New York, John Wiley and Sons.

\bibitem Box-Steffensmeier, Janet, John R. Freeman, Matthew P. Hitt, and
 Jon C. Pevehouse. 2014. \emph{Time Series Analysis for the Social Sciences}
 New York. Cambridge U

\bibitem Braithwaite, Alex and Shane D. Johnson. 2012 "Space-Time Modeling of
 Insurgency and Counterinsurgency in Iraq" \emph{Journal of Quantitative Criminology}
 28: 31-48.

 \bibitem Brandt, Patrick T., John R. Freeman and Philip Schrodt. 2014.
  "Evaluating Forecasts of Political Conflict Dynamics."
   \emph{International Journal of Forecasting} 30(4): 944-962.

 \bibitem Brandt, Patrick T., Michael Colaresi, and John R. Freeman 2008.
  "The Dynamics of Reciprocity, Accountability, and Credibility."
   \emph{Journal of Conflict Resolution} 52(3): 343-374.

 \bibitem Cameletti, Michela, Rosaria Ignaccolo, and Stefano Bande. 2011
  "Comparing Spatio-temporal models for Particulate Matter in the Piemonte."
   \emph{Environmentrics} 22: 985-996.

\bibitem Center for Army Analysis. Deployed Analyst History Report--Volume 1.
 Analytic Support to Combat Operations in Iraq 2002-2011. 2012 Fort Belvoir, Virginia.

 \bibitem Cseke, Botond, Andrew Zammit-Mangion, Tom Heskes, and Guido Sanguinetti. 2016.
  ``Sparse Approximate Inference for Spatio-Temporal Point Process Models."
  \emph{Journal of the American Statistical Association}" 111(516)" 1746-1763.

 \bibitem Cho, Wemdy K. "Contagion Effects and Ethnic Contribution Networks"
  \emph{American Journal of Political Science} 47(2): 368-387.

\bibitem Cho, Wendy K. and James G. Gimpel. 2003 "Prospecting for Gold"
  \emph{American Journal of Political Science} 51(2): 255-268.

  \bibitem Condra, Luke N. and Jacob N. Shapiro. 2012 "Who Takes the Blame?
 The Strategic Effects of Collateral Damage" \emph{American Journal of Political
 Science} 56(1): 167-187.

 \bibitem Donnay, Karsten and Vladimir Filmanov. 2014. "Views to a War: Systematic
  Differences in Media and Military Reporting of the War in Iraq"
  \emph{EPJ Data Science} 3 XXX

 \bibitem Enders, Walter. 2010. \emph{Applied Econometric Time Series} Third
   Edition. New York, Wiley.

 \bibitem Franzese, Robert Jr. , Jude C. Hays, and Scott J. Cook. 2015.
   "Spatial and Spatial-temporal-Autoregressive Probit Models of Interdependent
   Binary Outcomes." \emph{Political Science Research and Methods} 4(1): 151-173.

 \bibitem Franzese, Robert Jr. and Jude C. Hays. 2007.
  "Spatial-Econometric Models of Cross-Section Interdependence in Political Science
   Panel and Time Series Cross Section Data." \emph{Political Analysis} 15(2): 140-164.

   \bibitem Gill 2020.XXXX

 \bibitem Gneiting, Thomas. 2002 "Nonseparable, Stationary Covariance Functions
  for Space-Time Data" \emph{Journal of the American Statistical Association}
  97(458): 590-600

 \bibitem Hagan, John, Joshua Kaiser, and Anna Hanson. 2016. \emph{American Sociological
 Review} 81(2): 316-346.

 \bibitem Hammond, Jesse and Nils B. Weidman. 2014. "Using Machine Coded Even Data
  for the Micro-level Study of Political Violence" \emph{Reaearch and Politics}
  July-September: 1-6.

\bibitem Imai,Kosuke, and InSong Kim 2020. "On the Use of Two-Way Fixed Effects Regression
 Models for Causal Inference with Panel Data."
 \emph{Political Analysis} 29: 405-415

\bibitem Imai, Kosuke, and InSong Kim 2019 "When Should We Use Unit Fixed Effects
 Regression Models for Causal Inference with Longitudinal Data?"
 \emph{American Journal of Political Science} 63(2): 467-490.

 \bibitem Ingebrigtsen, R., F. Lindgren, I. Steinsland, and S. Martino. 2015.
  ``Estimation of a Nonstationary Model of for Annual Participation in Southern
   Norway Using Replicates of the Spatial Field." \emph{Spatial Statistics}C 14: 338-364.

 \bibitem Harvill, Jane L. 2010 "Spatio-temporal Processes."
  wires.wiley. com/compstats. vol 2: 376-382

 \bibitem Jun, and Scott Cook 2020.

 \bibitem Keele, Luke and Rocio Titiunik. 2015. "Geographic Boundaries as
  Regression Discontinuities." \emph{Political Analysis} 23: 127-155.

 \bibitem Keele, Luke and William Minozzi. 2013. "How Much is Minnesota
  Like Wisconain? Assumptions and Inference in Observational Data."
  \emph{Political Analysis} 21(2): 193-216

 \bibitem Linke, Andrew M., Frank W. Witmer, and John O`Loughlin. 2012.
  "Space Time Granger Analysis of the War in Iraq: A Study of Coalition
  and Insurgent Action-Reaction" \emph{International Interactions} 38: 402-425.

  \bibitem Lindgren, F. and H. Rue. 2015. "Bayesian Spatial and Spatial-Temporal
   Modeling with R-INLA." \emph{Journal of Statistical Software} 63(19).

  \bibitem Lindgren, F., H. Rue, and J. Lindstrom. 2011. "An Explicit Link Between
   Gaussian fields and Gaussian Markov Random Fields. The Stochastic Differential
   Equation Approach (with Discussion)." \emph{Journal of the Royal Statistical
   Society} Series B 73(4): 423-498

 \bibitem Martinelli, D. and G. Geniaux. 2017 "Approximate Likelihood Estimation
   of Spatial Probit Models" \emph{Regional Science and Urban Economics}
   64: 30-45.

 \bibitem Monogan III, J.E. and Jeff Gill 2015. "Measuring State and District
   Level Ideology With Spatial Realignment." \emph{Political Science Research
    and Methods} 4(1): 97-121.

 \bibitem Pavia, Jose Manuel, Beatriz Larraz, and Jose Maria Montero. 2008.
   "Election Forecasts Using Spatiotemporal Models" \emph{Journal of the
   American Statistical Association} 103(483): 1050-1059

\bibitem Python, A., J.B. Illian, C.M. Jones-Todd, and M.Blangiardo. 2018.
  "A Bayesian Approach to Modeling Subnational Spatial Dynamics of World
   Wide Non-state Terrorism, 2010-2016." \emph{Journal of the Royal Statistical
   Society} Series A pps. 1-22.

\bibitem Python, A., J. Illian, C. Jones-Todd, and M. Blangiardo. 2017.
  "Explaining the Lethality of Boko Haram's Terrorist Attacks in Nigeria,
  2009-2014." In \emph{Bayesian Statistics in Action} R.Argento et al.
  eds. Springer.

 \bibitem Schutte, Sebastian and Karsten Donnay. 2014. "Matched Wake Analysis:
 Finding Causal Relationships in Spatiotemporal Data" \emph{Political Geography}
 41: 1-10.

 \bibitem Schutte, Sebastian and Nils Weidman. 2011.
  ``Diffusion Patterns of Violence in Civil Wars."
  \emph{Political Geography} 30: 143-152.

 \bibitem Shapiro, Jacob N. and Nils B. Weidman 2015. "Is the Phone Mightier
  Than the Sword? Cellphones and Insurgent Violence in Iraq" \emph{International
  Organization} 69: 247-274

  \bibitem Silverman, Daniel 2021. "Too Late to Apologize? Collateral Damage,
    Post-Harm Compensation, and Insurgent Violence in Iraq." \emph{International
    Organization} XX

  \bibitem Stundal, Logan, Benjamin E. Bagozzi, John R. Freeman, and Jennifer S. Holmes.
    forthcoming. "Human Rights in Space: Statistical Modelss of Machine Coded vs. Human Coded Data."
    \emph{Political Analysis}


 \bibitem vonBorzykowski, Inkenand Michael Wahman. 2021  "Systematic Measurement Error
  in Election Violence Data: Causes and Consequences." \emph{British Journal of Political
  Science} XXX

  \bibitem Ward, Michael D. and Kristian Skrede Gleditsch. 2019 \textit{Spatial Regression
  Models} Second Edition. Los Angeles California. Sage Publications.

 \bibitem Ward, Michael D. and Kristian Skrede Gleditsch. 2002. "Location, Location,
   Location. A MCMC Approach to Modeling the Spatial Context of War and Peace."
   \emph{Political Analysis} 10(3): 244-260.

 \bibitem Weidman, Nils B. and Michael Ward. 2010. "Predicting Conflict in Space and
   Time" 2010. \emph{Journal of Conflict Resolution} 54(6): 883-901


 \bibitem Zammit-Mangion, Andrew, Michael Dewar, Visakan Kadirkamanathan, and Guido
  Sanguinetti. 2012. "Point Process Modeling of the Afgan War Diary". \emph{PNAS}
  109(31): 12414-12419.




\end{document}
